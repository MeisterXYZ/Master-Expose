% Muster f�r eine Bachelorarbeit im Studiengang Angewandte Mathematik an der 
% Fakult�t Informatik, Mathematik und Naturwissenschaften an der HTWK Leipzig 
% Stand: Oktober 2011, Prof. Dr. Martin Gr�ttm�ller
% Hinweis: Andere Kollegen k�nnen anderen Anspr�che an die Gesatltung haben.


\documentclass[a4paper,12pt]{scrartcl}

\usepackage{makeidx}
\usepackage{amsmath,amsfonts,amsthm}
%\usepackage{german}
\usepackage[ngerman]{babel}
\usepackage[square,numbers]{natbib}
\bibliographystyle{dinat}
\usepackage{tocstyle}
\usetocstyle{allwithdot}

\usepackage{graphicx}
\usepackage[latin1]{inputenc}
\usepackage[headsepline]{scrpage2}
\usepackage[
	    colorlinks=true,
	    urlcolor=blue,
	    citecolor=blue,
	    linkcolor=black
]{hyperref}
\usepackage{enumitem}


\sloppy
% Umgebungen (Dokumentation in amsthdoc.pdf):
\newtheorem{Satz}{Satz}[subsection]
\newtheorem{Lemma}[Satz]{Lemma}
\newtheorem{Korollar}[Satz]{Korollar}

\theoremstyle{definition}
\newtheorem{Definition}[Satz]{Definition}
\newtheorem*{Bemerkung}{Bemerkung}



% Beginn des eigentlichen Dokuments --------------------------------------------
\begin{document}

\ohead{\pagemark}
\ihead{\sectionmark}
\cfoot{}
\pagestyle{scrheadings}
\renewcommand{\sectionmark}[1]{\markright{#1}{}}
\thispagestyle{empty} % no page number

\begin{titlepage}
\begin{center}

\includegraphics[height=3cm]{HTWK_Logo_RGB}
% Titel ------------------------------------------------------------------------
\vspace*{1cm}

\bigskip
\bigskip
\large
{ \Large Expos� f�r eine Masterarbeit}

\bigskip
\small zur Erlangung des akademischen Grades\\[0.6cm]
\normalsize Master of Science\\[0.6cm]
\small im Studiengang Informatik\\
\small der Fakult�t Informatik und Medien\\
\small der Hochschule f�r Technik, Wirtschaft und Kultur Leipzig

\bigskip
\bigskip

\bigskip
\bigskip
{\Large Titel Ihrer Arbeit}\\
{\Large -- Untertitel --}

\end{center}

\bigskip
\bigskip
\large
\begin{tabbing}
\hspace{4cm}\=\kill
\footnotesize Vorgelegt von:  \> \small Raphael Drechsler\\
\footnotesize Anschrift:  \> \small Kieler Str. 34, 04357 Leipzig\\
\footnotesize Kontaktdaten:  \> \small Tel.: +49 1525 4194262  E-Mail: raphael.drechsler@googlemail.com\\
\footnotesize Matrikelnummer:  \> \small Raphael Drechsler\\
\footnotesize Fachsemester:  \> \small Raphael Drechsler\\
\small \\
\footnotesize Erstgutachter:  \> \small Prof. Dr.-Ing. Thomas Kudrass (HTWK)\\ 
\footnotesize Kontaktdaten:  \> \small Tel.: +49 341 3076-6420  E-Mail: thomas.kudrass@htwk-leipzig.de\\
\small \\
\footnotesize Zweitgutachter: \>  \small ... Torsten B�ttcher (integration-factory GmbH \& Co. KG)\\
\footnotesize Kontaktdaten:  \> \small Tel.: +49 69 25669269-0  E-Mail: boettcher@integration-factory.de\\
\small \\
\footnotesize Vorauss. Abgabedatum \> \small Januar 2020\\
\footnotesize Datum \> \small \today\\

\end{tabbing} 
\end{titlepage}

% Inhaltsverzeichnis ------------------------------------------------------------------------
\clearpage
\tableofcontents
\clearpage

% Einleitung ------------------------------------------------------------------------
\clearpage
\section{Problemstellung}
Das Unternehmen \textit{integration-factory} ist ein Consulting-Unternehmen und bietet f�r seine Kunden L�sungen in den Bereichen Business Intelligence und insbesondere Integration von Unternehmensdaten an.\\
Gegenw�rtig besteht dabei in zwei Kunden-Projekten eine Situation, in der durch \textit{integration-factory} eine DWH-L�sung bereitgestellt wurde, welche von einer Marktanalyse von Scheduling-Tools sowie einer Betrachtung von ETL-Worklflow-Optimierung profitieren k�nnte.\\
Im Folgenden sollen diese zwei Kundenszenarios beschrieben werden.

\paragraph{Kundenszenario A}
In Kunden-Solution A ist bereits das kommerzielle ETL-Scheduling-Tool \textit{Control-M} von \textit{BMC Software} im Einsatz. Mit den derzeit im Projekt umgesetzten ETL-Workflows treten Probleme mit unn�tigen Wartezeiten bei Status�berg�ngen im Workflow auf. Es besteht der Bedarf an einer performanteren L�sung.\\
Durch die Umsetzung dieser Arbeit sollen f�r diese Kundensituation entsprechende Handlungsoptionen evaluiert werden. Dies soll einerseits durch das Untersuchen der Optimierungsm�glichkeiten in \textit{Control-M} sowie der Betrachtung von Workflow-Optimierungen und andererseits durch einen Vergleich der \textit{Control-M}-L�sung mit weiteren Scheduler-Alternativen geschehen.\\

\paragraph{Kundenszenario B}
In Kundensolution B kein dediziertes Scheduling-Tool im Einsatz, stattdessen ETL-Tool informatica.\\
Tool liefert nur rudiment�re Funktionalit�ten zum Scheduling. \\
Bedarf L�sung �ber Integration von Scheduler zu verbessern, um mehr M�glichkeiten zu haben komplexere ETL-Workflows mit dem Tool l�sen zu k�nnen.


% Zielsetzung und Erkenntnisse ------------------------------------------------------------------------
\section{Zielsetzung}
Insgesamt also:


% Vorgehensweise ------------------------------------------------------------------------
\section{Vorgehensweise}
\subsection{Vorbereitung Evaluation}
- Untersuchung Themenfeld ETL-Scheduler und ableiten von Anforderungen an einen ETL-Scheduler\\
- Ableiten weiterer Anforderungen, welche sich aus spezieller Kundensituation ergeben.\\
- Aufstellen von Bewertungskriterien und Bewertungsma�st�ben\\
- Aus Kundenszenarios repr�sentative Referenz-Workflows ableiten anhand derer sp�ter Performancemessung durchgef�hrt wird. Ggf. werden dabei oder in einem sp�teren Schritt Optimierungsm�glichkeiten f�r die bestehenden Workflows sichtbar.
\subsection{Kandidaten suchen}
- Suchen von Commerce und Open-Source-Scheduler-Tools und Vorstellung der Kandidaten in schriftlicher Ausarbeitung\\
- KO-Check: Ist das jeweilige Tool in die Kundensolution integrierbar und erf�llt es die KO-kriterien?
\subsection{Durchf�hrung Evaluation}
- Umsetzung der Referenz-Workflows und Performancemessungen in den verschiedenen Schedulern\\
- Suchen von Optimierungsm�glichkeiten in einzelnen L�sungen\\
-  Bewertung der �brigen Kriterien in der Bewertungsmatrix\\
- Auswerten der Bewertungsmatrix
\subsection{Formulieren der Konsequenzen f�r Kundensolution A und B}
Formulieren der Konsequenzen welche sich aus dem Ergebnis der Evaluation ergeben.\\
Ggf. erfolgt eine Umsetzung der Beschriebenen Anpassung/�nderungen beim Kunden.

% Voraussichtliche Gliederung ------------------------------------------------------------------------
\section{Voraussichtliche Gliederung}
Aus der geschilderten Vorgehensweise ergibt sich die folgende voraussichtliche Gliederung.

\begin{enumerate}
	\item Topic
	\begin{enumerate}[label*=\arabic*.]
		\item First Subtopic
		\item Second Subtopic
		\begin{enumerate}[label*=\arabic*.]
			\item First Sub-Subtopic
			\item Second Sub-Subtopic
		\end{enumerate}
	\end{enumerate}
\end{enumerate}


\section{Erste Literaturverweise}

Generelle Literatur zu DWH\\
Panos Paper zu Einstieg in Literatur mit Parallelisierung und Sequentiallisierung\\
Vergleich Features von Anbietern an den Stellen wo Literatur fehlt.

\section{Zeitplan}
gegenw�rtig offenes To-Do.

% ***********************************************
\end{document}
% ***********************************************

